%% LyX 2.3.6.1 created this file.  For more info, see http://www.lyx.org/.
%% Do not edit unless you really know what you are doing.
\documentclass[spanish,english]{article}
\usepackage[T1]{fontenc}
\usepackage[latin9]{inputenc}
\usepackage{amssymb}
\usepackage{graphicx}

\makeatletter

%%%%%%%%%%%%%%%%%%%%%%%%%%%%%% LyX specific LaTeX commands.
\newcommand{\noun}[1]{\textsc{#1}}

\makeatother

\usepackage{babel}
\addto\shorthandsspanish{\spanishdeactivate{~<>}}

\begin{document}
\title{Tarea Examen 1. Modulo I Python.}
\maketitle

\section*{Ejercicio 1.}

Hay $N$ puntos (enumerados de $0$ a $N-1$) en un plano. Cada punto
es de color rojo ('R') o verde ('G'). El punto $k$-�simo est� ubicado
en las coordenadas $(X[k],Y[k])$ y su color es $colors[k]$. Ning�n
punto se encuentra en las coordenadas $(0,0)$.

Queremos dibujar un c�rculo centrado en las coordenadas $(0,0)$,
tal que el n�mero de puntos rojos y verdes dentro del c�rculo sea
igual. �Cu�l es el n�mero m�ximo de puntos que pueden se pueden encerrar
por uno de estos c�rculos centrados en el origen? Tenga en cuenta
que siempre es posible dibujar un c�rculo sin puntos en el interior.

Este ejercicio consiste en escribir una funci�n en python
\begin{itemize}
\item def sol ($X$, $Y$, $colors$)
\end{itemize}
tal que dados dos arreglos de enteros $X$, $Y$ (de tama�o $N$)
y una cadena $colors$, regresa un entero especificando el m�ximo
n�mero de puntos dentro de un c�rculo que contiene un n�mero igual
de puntos rojos y puntos verdes. 

El algoritmo debe cumplir las siguientes suposiciones:
\begin{itemize}
\item $N$ es un entero entre $1$ y $100000$;
\item Cada elemento de los arreglos $X$ y $Y$ es un entero dentro del
intervalo cerrado $[-20000,20000]$;
\item El string colors consiste en solo $N$ caracteres ``R'' y/o ``G'';
\item No hay puntos en el origen $(0,0)$. 
\end{itemize}
\noun{Ejemplo.} 

Dado los arreglos $X=[4,0,2,-2]$, $Y=[4,1,2,-3]$ y $colors="RGRR"$
tu funci�n debe regresar 2. El c�rculo contiene los puntos $(0,1)$
y $(2,2)$ pero no los puntos $(-2,-3)$ y $(4,4)$.
\begin{center}
\includegraphics{\string"Im�genes/Captura de pantalla de 2022-09-22 23-19-59\string".eps}
\par\end{center}

\section*{Ejercicio 2.}

Definamos el conjunto de los car�cteres alfanum�ricos

\begin{equation}
\mathcal{L}=\{\emptyset,a,b,c,d,\ldots,z,A,B,C,\ldots,Z,0,1,\ldots,9,\>\}\>,
\end{equation}
en donde el �ltimo elemento trata de representar el espacio que se
utiliza en cualquier texto. 

Sobre el conjunto $\mathcal{L}$ se puede definir una m�trica determinada
de la siguiente manera:

\begin{equation}
d_{1}(x,y)=\left\{ \begin{array}{ccc}
1 & \>si & x=y\\
\\
0 & si & x\neq y
\end{array}\right.\>.
\end{equation}
Demuestre entonces que el espacio $\mathcal{L}$ es en efecto un espacio
m�trico con la funci�n $d_{1}(x,y)$, esto significa que:

\selectlanguage{spanish}%
\begin{equation}
\begin{array}{ccccc}
d_{1}(x,y) &  & = &  & d_{1}(y,x)\\
\\
d_{1}(x,y)=0 &  & \Longleftrightarrow &  & x=y\\
\\
d_{1}(x,z) &  & \leq &  & d_{1}(x,y)+d_{1}(y,z)
\end{array}
\end{equation}
para todo $x,y,z\in\mathcal{L}$.

Sugerencia: para demostrar la desigualdad del tri�ngulo h�galo por
casos (CASO 1 $x=y$ y $y=z$, CASO 2 $x=y$ y $y\neq z$, etc ).
\selectlanguage{english}%

\section*{Ejercicio 3.}

Definimos el conjunto que contiene a todas las palabras:

\begin{equation}
WordSpace=\mathcal{L}\times\mathcal{L}\times\mathcal{L}\times\cdots
\end{equation}
Los elementos de este conjunto son n-adas tales que si $x\in WordSpace$
entonces

\begin{equation}
\begin{array}{ccc}
x_{n}\in\mathcal{L} &  & \forall n\in\mathbb{N}\end{array}
\end{equation}
Con base en esto definimos la funci�n 

\begin{equation}
d(x,y)\equiv\sum_{n=1}^{\infty}d_{1}(x_{n},y_{n})
\end{equation}
Demuestre que $d(x,y)$ es una m�trica y tambi�n demuestre que existe
una biyecci�n entre $\mathbb{R}$ y $WordSpace$. Concluya entonces
cu�l es la cardinalidad de $WordSpace$.

\selectlanguage{spanish}%
Sugerencia: para demostrar la desigualdad del tri�ngulo h�galo por
casos (CASO 1 $d(x,y)$ es infinito y $d(y,z)$ es infinito, CASO
2 $d(x,y)$ es finito y $d(y,z)$ es infinito, etc ).\selectlanguage{english}%

\end{document}
